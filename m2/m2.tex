\documentclass{article}\usepackage{amsmath,amssymb,amsthm,tikz,tkz-graph,color,chngpage,soul,hyperref,csquotes,graphicx,floatrow, listings,polynom}\newcommand*{\QEDB}{\hfill\ensuremath{\square}}\newtheorem*{prop}{Proposition}\renewcommand{\theenumi}{\alph{enumi}}\usepackage[shortlabels]{enumitem}\usepackage[nobreak=true]{mdframed}\usetikzlibrary{matrix,calc}\MakeOuterQuote{"}\usepackage[margin=0.75in]{geometry} \newtheorem{theorem}{Theorem}\newcommand{\Z}{\mathbb Z}\newcommand{\R}{\mathbb R}\newcommand{\Q}{\mathbb Q}\newcommand{\N}{\mathbb N}\newcommand{\x}[1]{\textrm{ #1 }}\newcommand{\pr}{\textrm{Pr}}
\newcommand{\dincludegraphics}{\includegraphics[width=0.5\textwidth]}
\newcommand{\tincludegraphics}{\includegraphics[width=0.33\textwidth]}
\newcommand{\sumlim}[3]{\sum\limits_{#1}^{#2}#3}
\newcommand{\eq}[1]{\begin{equation}#1\end{equation}}

\title{CS70 - Midterm 2 Notes}
\author{Name: Felix Su$\quad$SID: 25794773}
\date{Spring 2016$\quad$GSI: Gerald Zhang}
\begin{document}
\maketitle

%%%% Topic %%%%
\subsection*{Review}
%%%% Notes %%%%
\begin{itemize}
    \item $e$ and $(p - 1)(q - 1)$ are \textbf{Relatively Prime}: $gcd(e,(p - 1)(q - 1)) = 1$
    \item $d$ is the \textbf{Multiplicative Inverse} of $e \bmod (p-1)(q-1)$: $ed = 1 \bmod (p-1)(q-1)$
    \item If $p$ is prime, \textbf{Fermat's Little Thm}: $x^{p-1} = 1 \bmod p$
\end{itemize}
%%%% Fast Exponentiation Example %%%%
\begin{mdframed}
\textbf{Fast Exponentiation Example}
$2^{17} \bmod 35$\\
$2^2 \equiv 4	\bmod	35$\\
$2^4 \equiv 4^2 \equiv 16 \bmod 35$\\
$2^8 \equiv 16^2 \equiv 11 \bmod 35$\\
$2^{16} = 11^2 \equiv 16 \bmod 35$\\
$2^{17} \equiv 2^{16+1} \equiv 2^{16} \cdot 2 \equiv 16\cdot2 \bmod 35 \equiv 32$
\end{mdframed}
%%%% Euclid's Extended Algorithm %%%%
\begin{mdframed}
\textbf{Euclid's Extended Algorithm: $egcd(x,y)$}
\begin{lstlisting}[language=Python, mathescape=true]
# precondition: Assumes $x \ge y \ge 0$ and $x > 0$
# postcondition: Outputs $(d,a,b)$ where $d = gcd(x, y)$ and $a,b \in \Z$ with $d = ax+by$
egcd$(x, y)$:
    if $(y == 0)$:
        return $(x,1,0)$
    else:
        let $(d,a,b) = egcd(y, x \pmod{y})$
        return $(d, b, a-\lfloor x/y \rfloor b)$
\end{lstlisting}
\textbf{Example}
$45 \equiv 37^{-1} \pmod{64}$\\
$egcd(64, 37) : (d,a,b) = (1, 11,  -8 - 1(11) = -19)$\\
$egcd(37, 27) : (d,a,b) = (1, -8, 3 - 1(-8) = 11)$\\
$egcd(27, 10) : (d,a,b) = (1, 3, -2 - 2(3) = -8)$\\
$egcd(10, 7) : (d,a,b) = (1, -2, 1 - 1(-2) = 3)$\\
$egcd(7, 3) : (d,a,b) = (1, 1, 0 - 2(1) = -2)$\\
$egcd(3, 1) : (d,a,b) = (1, 0,  1 - 3(0) = 1)$\\
$egcd(1, 0) : (d,a,b) = (1, 1, 0)$\\
$1 = 11(64) + -19(37)$\\
$-19(37) \equiv 1 \pmod{64}$\\
$64 - 19 = 45$
\end{mdframed}
%%%% Binomial Theorem %%%%
\begin{mdframed}
\textbf{Binomial Theorem}
\eq{(a+b)^n=\sumlim{k=0}{n}{\binom{n}{k}a^{n-k}b^k}}
\end{mdframed}
%%%% Binomial Formula %%%%
\begin{mdframed}
\textbf{Binomial Formula}
\eq{\binom{n}{k}=\frac{n!}{(n-k)!k!}}
\end{mdframed}

P(x) = y1∆1(x) +y2∆2(x) +···+yd+1∆d+1(x)

Graphs

Error-correcting Codes
List Decoding

Independence
Bonferroni Inequalities
Sample Space
Union Bound
Unlikely Probability
Union
Intersection
Collisions
Probability Space
Approximate Limit

Counting
Countability

% 	Finite Fields
% 	Degree
% 	Divides
% Berlekamp-Welsh

%%%%%%%%%%%%%%%%%%%%%%%%%%%%%%%%%%%%%
 \pagebreak
%%%%%%%%%%%%%%%%%%%%%%%%%%%%%%%%%%%%%

%%%% Topic %%%%
\subsection*{RSA Encryption}
%%%% Notes %%%%
\begin{itemize}
    \item Public Key Encryption 2 keys:
    \begin{itemize}
        \item Public key: $(N, e)$ known to everyone, Encryption Function: $E(x) \equiv x^e \bmod N$
        \item Private key: $d$ only known to Bob, Decrypting Function: $D(x) \equiv x^d \bmod N$
    \end{itemize}
    \item Bijection that can only be inverted with private key
    \item Sender applies the encryption function $E$ to $x$ to obtain a ciphertext $E(x)$, which she then sends to receiver ($E : \{0,...,N-1\} \mapsto \{0,\cdots,N-1\}$) ($x \in \{1,\cdots,N-1\}$)
    \item Receiver applies decryption function $D$ to recover x.
    \item $D(E(x)) = x \bmod N$ for all $x \in \{0,1,\cdots,N-1\}$
    \begin{itemize}
        \item $(x^e)^d = x \bmod N$ for every $x \in \{0,1,...,N-1\}$
    \end{itemize}
    \item Key creator has to find prime numbers $p$ and $q$, each having many (say, 512) bits.
    \item Sender has to compute $y=E(x)=x^e \bmod N$, and Receiver has to compute $D(y)= y^d \bmod N$
\end{itemize}
\begin{mdframed}
\textbf{Given}
\begin{itemize}
    \item $N = pq$ for two large primes $p$ and $q$, usually a 512-bit number
    \item $e$ is relatively prime to $(p-1)(q-1)$
    \item $d$ = multiplicative inverse of $e \bmod (p-1)(q-1)$
\end{itemize}
\eq{E(x) \equiv x^e \bmod N}
\eq{D(x) \equiv x^d \bmod N}
\end{mdframed}

%%%%%%%%%%%%%%%%%%%%%%%%%%%%%%%%%%%%%
\pagebreak
%%%%%%%%%%%%%%%%%%%%%%%%%%%%%%%%%%%%%

%%%% Topic %%%%
\subsection*{Counting}
%%%% Notes %%%%
\textbf{Combinations and Permutations}
\begin{itemize}
    \item Order
    \begin{itemize}
        \item Matters:
        \item Doesn't Matter:
    \end{itemize}
    \item Replacement
    \begin{itemize}
        \item With:
        \item Without:
    \end{itemize}
\end{itemize}
\begin{mdframed}
\textbf{Highlight:}
\end{mdframed}

%%%% Topic %%%%
\subsection*{Halt Program}
%%%% Notes %%%%
\textbf{Reduction}
\begin{itemize}
    \item Disprove program's existance by reducing to "I can use this program to solve HALT"
\end{itemize}
\begin{mdframed}
\textbf{Highlight:}
\end{mdframed}
%%%%%%%%%%%%%%$
\textbf{Diagonalization}
\begin{itemize}
    \item Listing
\end{itemize}
\begin{mdframed}
\textbf{Highlight:}
\end{mdframed}
%%%%%%%%%%%%%%$
\textbf{Turing Program}
\begin{itemize}
    \item 
\end{itemize}
\begin{mdframed}
\textbf{Highlight:}
\end{mdframed}

%%%% Topic %%%%
\subsection*{Probability}
%%%% Notes %%%%
\textbf{Bayes' Rule}
\begin{itemize}
    \item Content
\end{itemize}
\begin{mdframed}
\eq{\pr[A|B]=\frac{\pr[A]\pr[B|A]}{\pr[B]}}
\textbf{Uses}
\begin{itemize}
    \item Find probability of $A$ given $B$
    \item Use Law of Total Probability to find $\pr[A]$ and $\pr[B]$ if prior probabilities exist
\end{itemize}
\end{mdframed}
%%%%%%%%%%%%%%%
\textbf{Product Rule}
\begin{itemize}
    \item Content
\end{itemize}
\begin{mdframed}
\eq{\pr[A_1\cap\cdots\cap A_m]=\pr[A_1]\pr[A_2|A_1]\cdots\pr[A_m|A_1\cap\cdots\cap A_{m-1}]}
\end{mdframed}
%%%%%%%%%%%%%%%
\textbf{Law of Total Probabilities}
\begin{itemize}
    \item Probability of event $B$ is equivalent to the sum of the product of each of $B$'s prior probabilities and the chance of $B$ occuring given that prior probability.
\end{itemize}
\begin{mdframed}
\eq{\pr[B]=\pr[A_1]\pr[B|A_1]+\cdots+\pr[A_n]\pr[B|A_n]}
\textbf{Uses}
\begin{itemize}
    \item Find denominator for Bayes' Rule problem
    \item Get probability of any event that has prior probabilities
\end{itemize}
\end{mdframed}
%%%%%%%%%%%%%%%
\textbf{Bonferroni's Inequalities}
\begin{itemize}
    \item Content
\end{itemize}
\begin{mdframed}
\eq{\pr[A \cap B] > \pr[A] + \pr[B] - 1}
\eq{\pr[A_1 \cap\cdots\cap A_n] > \pr[A_1] + . . . + \pr[A_n] - (n - 1)}
\textbf{Uses}
\begin{itemize}
    \item Stuff
\end{itemize}
\end{mdframed}

\end{document}