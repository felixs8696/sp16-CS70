\documentclass{article}\usepackage{amsmath,amssymb,amsthm,tikz,tkz-graph,color,chngpage,soul,hyperref,csquotes,graphicx,floatrow, listings}\newcommand*{\QEDB}{\hfill\ensuremath{\square}}\newtheorem*{prop}{Proposition}\renewcommand{\theenumi}{\alph{enumi}}\usepackage[shortlabels]{enumitem}\usepackage[nobreak=true]{mdframed}\usetikzlibrary{matrix,calc}\MakeOuterQuote{"}\usepackage[margin=0.75in]{geometry} \newtheorem{theorem}{Theorem}\newcommand{\Z}{\mathbb Z}\newcommand{\R}{\mathbb R}\newcommand{\Q}{\mathbb Q}\newcommand{\N}{\mathbb N}\newcommand{\x}[1]{\textrm{ #1 }}\newcommand{\pr}{\textrm{Pr}}
\newcommand{\dincludegraphics}{\includegraphics[width=0.5\textwidth]}
\newcommand{\tincludegraphics}{\includegraphics[width=0.33\textwidth]}
\newcommand{\sumlim}[3]{\sum\limits_{#1}^{#2}#3}
\newcommand{\eq}[1]{\begin{equation}#1\end{equation}}

\title{CS70 - Combinatorial Arguments}
\author{Name: Felix Su$\quad$SID: 25794773}
\date{Spring 2016$\quad$GSI: Gerald Zhang}
\begin{document}
\maketitle

%%%% Topic %%%%
\subsection*{Strategy}
%%%% Notes %%%%
\begin{enumerate}[1.]
    \item Define equivalent quantity $Q$
    \item Count $Q$ in a way to determine $LHS=Q$
    \item Count $Q$ in a way to determine $RHS=Q$
    \item Conclude $LHS=RHS$
    \item \textbf{Review: 2,3,4,7,8,9}
\end{enumerate}
%%%% Topic %%%%
\subsection*{Hints}
%%%% Notes %%%%
\begin{itemize}
    \item $\binom{n}{k}\equiv\binom{n}{n-k}$: Include $k$ out of $n$ items = Do not include $n-k$ items
    \item $2^n$= All possible ways to choose any number of $n$ items. Counting Product Rule.
    \item $\sum\binom{n}{k}=\binom{n+1}{k+1} \rightarrow$ set one value to include and pick remaining.
    \item $\sum\binom{n}{k}=\binom{n+1}{k} \rightarrow$ set aside one value to not include and pick remaining.
\end{itemize}
%%%% Topic %%%%
\subsection*{Examples}
%%%% Notes %%%%
\begin{mdframed}
\begin{equation}2^n=\sumlim{k=0}{n}{\binom{n}{k}}\end{equation}
\textbf{LHS:} Using counting product rule, this is the number subsets of $n$ decided by either choosing or not choosing each item. There are a total of $2^n$ possible subsets of size $k \in (0,n)$ from a total of $n$ items.\\\\
\textbf{RHS:} The summation of all possible ways to choose $k \in (0,n)$ items from $n$, which is the same as left.
\end{mdframed}
\begin{mdframed}
\begin{equation}\sumlim{m=k}{n}{\binom{m}{k}}=\binom{n+1}{k+1}\end{equation}
\textbf{LHS:} For each iteration of the summation, set aside a largest element $l$ then choose the remaining $k$ elements to obtain a subset of $n$ that is size $k+1$. Such that you only pick items less than $l$, limiting you to $m$ items to choose from. All combinations of this is the same as RHS.\\\\
\textbf{RHS:} All possible subsets of $\{1,\cdots,n+1\}$ that are size $k+1$.
\end{mdframed}
\begin{mdframed}
\eq{\sumlim{k=0}{n}{\binom{m+k}{k}}=\binom{n+m+1}{n}}
\textbf{LHS:} Specify the smallest element from $\{1,\cdots,n+m+1\}$ not in the selected subset. If it is 1, you must choose the remaining $n$ items from $\{2,\cdots,n+m+1\}$, $\binom{m+n}{n}$, which is the $k=n$ term. If it is 2, 1 is included in the subset, so you only need to choose from $n-1$ remaining items from $\{2,\cdots,n+m+1\}$, $\binom{m+n-1}{n-1}$. This continues down to having the smallest number not in the set be $n+1$, which would cause, $\binom{m+0}{0}$. This equates to all possible ways to select select $n$ items from $n+m+1$ choices\\\\
\textbf{RHS:} All possible ways to select select $n$ items from $n+m+1$ choices.
\end{mdframed}
\begin{mdframed}
\eq{\sumlim{k=0}{n}{\binom{n}{k}^2}=\binom{2n}{n}}
\textbf{LHS:} Split $\{\pm1,\cdots,\pm n\}$ into 2 sets, 1 of positive and 1 of negative elements. Choose $k$ positive items and $n-k$ negative items (which is still equivalent to $\binom{n}{k}$). This is equivalent to choosing $n$ items out of $\{\pm1,\cdots,\pm n\}$.\\\\
\textbf{RHS:} All possible ways to directly choose $n$ items out of $\{\pm1,\cdots,\pm n\}$.
\end{mdframed}
\begin{mdframed}
\eq{\sumlim{k=0}{r}{\binom{m}{k}\binom{n}{r-k}}=\binom{n+m}{r}}
\textbf{LHS:} Split the collection of books into a set of textbooks and one of comic books. Select $k$ books from the set of textbooks and $r-k$ books from the set of $m$ comic books. This is the same as selecting $r$ total books from a collection of $n$ textbooks and $m$ comic books.\\\\
\textbf{RHS:} Choose $r$ books from a collection of $n$ textbooks and $m$ comic books.
\end{mdframed}
\begin{mdframed}
\eq{\binom{n}{k}=\binom{n-1}{k-1}+\binom{n-1}{k}}
\textbf{LHS:} All possible ways to select $k$ items from a collection of $n$ choices.\\\\
\textbf{RHS:} Set aside one element. Assuming this item must be in the selected subset of $k$ items, select the remaining $k-1$ items from the remaining $n-1$ choices $\binom{n-1}{k-1}$. Then assume the item set aside is not in the selected subset of $k$ items. Select the rest of the $k$ items from the remaining $n-1$ choices $\binom{n-1}{k}$. These two disjoint sets add up to the original selection of $k$ items from $n$ choices
\end{mdframed}
\begin{mdframed}
\eq{\binom{2n}{2}=2\binom{n}{2}+n^2}
\textbf{LHS:} All possible ways to choose 2 items from a set of $2n$ choices.\\\\
\textbf{RHS:} There are 3 ways to choose $2$ items from a set of $2n$ choices if you split the $2n$ choices into 2 sets of $n$ choices. You can choose both from the first set of $n$ choices $\binom{n}{2}$, choose both from the second set of $n$ choices $\binom{n}{2}$, or you can choose 1 from each set $\binom{n}{1}^2=n^2$. Add these disjoint sets together to get LHS.
\end{mdframed}
\begin{mdframed}
\eq{\sumlim{k=0}{n}k\binom{n}{k}=n2^{n-1}}
\textbf{LHS:} Pick a team of size $k$, then pick the captain out of the $k$ people on the team. All possible sizes of the team is $k=(0,n)$, so sum all those possibilities.\\\\
\textbf{RHS:} Pick one of $n$ people to be a captain. For each rest of the $n-1$ people, choose whether or not to include him/her on the team. (Counting)
\end{mdframed}
\begin{mdframed}
\eq{\sumlim{k=j}{n}\binom{n}{k}\binom{k}{j}=2^{n-j}\binom{n}{j}}
\textbf{LHS:} Select a team of size $k$ from $n$ students, then choose $j$ leaders from the $k$ team members. Sum all the probabilities for teams of size $j$ to $n$ \\\\
\textbf{RHS:} Out of $n$ students, choose $j$ leaders, out of the remaining $n-j$ people, select whether or not each will be on the team.
\end{mdframed}
\begin{mdframed}
\eq{\sumlim{i=0}{k}\binom{m}{i}\binom{n}{k-i}=\binom{m+n}{k}}
\textbf{LHS:} Split a collection of $m$ fiction and $n$ non-finction books into a set of $m$ fiction books and $n$ non-fiction books. Select $k$ total books by selecting $i$ books from the fiction pile and the remaining $k-i$ from the non-finction pile. This is the same as RHS.\\\\
\textbf{RHS:} All possible ways to directly choose $k$ books from a collection of $m$ fiction and $n$ non-fiction books.
\end{mdframed}
\end{document}