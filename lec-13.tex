\documentclass{article}\usepackage{amsmath,amssymb,amsthm,tikz,tkz-graph,color,chngpage,soul,hyperref,csquotes,graphicx,floatrow, listings}\newcommand*{\QEDB}{\hfill\ensuremath{\square}}\newtheorem*{prop}{Proposition}\renewcommand{\theenumi}{\alph{enumi}}\usepackage[shortlabels]{enumitem}\usepackage[nobreak=true]{mdframed}\usetikzlibrary{matrix,calc}\MakeOuterQuote{"}\usepackage[margin=0.75in]{geometry} \newtheorem{theorem}{Theorem}\newcommand{\Z}{\mathbb Z}\newcommand{\R}{\mathbb R}\newcommand{\Q}{\mathbb Q}\newcommand{\N}{\mathbb N}

\title{CS70 - Lecture 13 Notes}
\author{Name: Felix Su$\quad$SID: 25794773}
\date{Spring 2016$\quad$GSI: Gerald Zhang}
\begin{document}
\maketitle

%%%% Topic %%%%
\subsection*{Countability Summary:}
%%%% Notes %%%%
\begin{itemize}
    \item $S$ is countable if there is a bijection between $S$ and some subset of $\N$.
    \item If the subset of $\N$ is finite, $S$ has finite cardinality.
    \item If the subset of $\N$ is infinite, $S$ is countably infinite.
    \item Bijection with natural numbers $\implies$ countably infinite.
    \item Enumerable (listing) $\equiv$ countable.
    \item Subset of countable set is countable.
    \item All countably infinite sets have the same cardinality
    \item Natural numbers have finite number of digits
\end{itemize}

%%%% Topic %%%%
\subsection*{Digonalization:}
%%%% Notes %%%%
\begin{itemize}
    \item The set of all subsets $S_i$ of $\N$ (powerset of $\N$ is not countable
    \begin{itemize}
        \item Arbitrary Listing: $L$
        \item Diagonal set $D$: For each index $i$ of $L$, if $i \not\in S_i$, put $i$ in $D$, otherwise omit $i$
        \item $D$ is not in $L$ by construction: $D$ is different from each $i$th set $S_i$ in L, for every $i$
        \item $D$ is a subset of $N$: every element in $D$ is a natural number
        \item $L$ does not contain all subsets of $N$: Contradiction
    \end{itemize}
\end{itemize}

\begin{mdframed}
\textbf{Diagonalization Algorithm:}
    \begin{enumerate}[1.]
        \item Assume that set $S$ can be enumerated.
        \item Consider an arbitrary list of all the elements of $S$.
        \item Use the diagonal from the list to construct a new element $t$.
        \item Show that $t$ is different from all elements in the list $\implies t$ is not in the list.
        \item Show that $t$ is in $S$.
        \item Contradiction.
    \end{enumerate}
\end{mdframed}

%%%% Topic %%%%
\subsection*{Cardinalities:}
%%%% Notes %%%%
\begin{mdframed}
\textbf{Continuum Hypothesis:}
\begin{itemize}
    \item Goedel proved this hypothesis cannot be proven with math we currently know
    \item There is no infinite set whose cardinality is between the cardinality of an infinite set and its power set.
\end{itemize}
\end{mdframed}
\pagebreak

\textbf{Uncountable Sets:}
\begin{itemize}
    \item Prove equivalence between cardinalities
    \item Show bijection exists between two sets: uncountable sets cannot be enumerated
    \item Create function $f: B \rightarrow A$ (can include multiple casesfor certain domains)
    \item Prove mapping is one to one by testing on arbitrary values: $x, y$ (Need to validate for multiple cases)
    \begin{itemize}
        \item Example: $|[0,1]| \equiv |\R|$
        \item $f: \R^+ \rightarrow [0,1]$
    \end{itemize}
\end{itemize}

%%%% Topic %%%%
\subsection*{Undecidability:}
%%%% Notes %%%%
\begin{mdframed}
\textbf{Russell's Paradox:}
    \begin{itemize}
        \item Naive Set Theory: Any definable collection is a set.
        \begin{equation}\exists y\forall x(x \in y \iff P(x))\end{equation}
        \item NST : $y$ = the set of elements that satifies $P(x)$
        \item Make statement: $P(x) = x \not\in x$
        \item By NST: There exists a $y$ that satisfies above statement for $P(x)$
        \item Plug in $x = y$ to NST
        \begin{equation}y \in y \iff y \not\in y\end{equation}
        \item Methematicl system is broken, because conditions and statements are false and contradictions
    \end{itemize}
\end{mdframed}

%%%% Topic %%%%
\subsection*{HALT: DNE}
%%%% Notes %%%%
\begin{itemize}
    \item $HALT(P,I)$ : P = program, I = input to program
    \begin{itemize}
        \item Theoretically determines if $P(I)$ halts or loops forever
    \end{itemize}
\end{itemize}

\begin{mdframed}
\textbf{Halt Turing Proof:}
\begin{itemize}
    \item Assume $HALT(P,I)$ exists
    \item Set $P = Turing(P)$
    \item Use Diagonalization
\end{itemize}
\begin{lstlisting}[language=Python]
    def Turing(P):
        if(HALT(P,P)): #halts
            go into infinite loop
        else
            halt immediately
\end{lstlisting}
\begin{itemize}
    \item Assume $Turing(Turing)$ halts
    \item Run $HALTS(Turing, Turing)$
    \begin{itemize}
        \item if 'halts', Turing(Turing) 'goes into infinite loop'
        \item if loops forever, Turing(Turing) 'halts immediately'
    \end{itemize}
    \item Contradiction, so $HALT(P,P)$ does not exist
\end{itemize}
\end{mdframed}

\begin{mdframed}
\textbf{Halt Diagonlization Proof:}
\begin{itemize}
    \item Program and input are both enumerable (fixed length strings)
    \item Program either halts or loops on any input
    \item Create list: $P_i \rightarrow {P_j(P)}$ where $i,j \in \N$ 
    \item Each entry of list is arbitrarily $HALT$ or $LOOP$
    \item Diagonal exists, so create $Turing()$ s.t. it returns opposite values along the diagonal
    \item This means $Turing()$ is not in the list $\implies Turing()$ is not a prgoram
    \begin{itemize}
        \item $Turing$ is a simple function constructed from $HALT$
        \item $\therefore\, Turing()$ DNE $\implies HALT()$ DNE
    \end{itemize}
\end{itemize}
\end{mdframed}

%%%% Topic %%%%
\subsection*{Undecidable Problems}
%%%% Notes %%%%
\begin{itemize}
    \item 
\end{itemize}
\end{document}