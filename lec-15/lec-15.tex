\documentclass{article}\usepackage{amsmath,amssymb,amsthm,tikz,tkz-graph,color,chngpage,soul,hyperref,csquotes,graphicx,floatrow, listings}\newcommand*{\QEDB}{\hfill\ensuremath{\square}}\newtheorem*{prop}{Proposition}\renewcommand{\theenumi}{\alph{enumi}}\usepackage[shortlabels]{enumitem}\usepackage[nobreak=true]{mdframed}\usetikzlibrary{matrix,calc}\MakeOuterQuote{"}\usepackage[margin=0.75in]{geometry} \newtheorem{theorem}{Theorem}\newcommand{\Z}{\mathbb Z}\newcommand{\R}{\mathbb R}\newcommand{\Q}{\mathbb Q}\newcommand{\N}{\mathbb N}\newcommand{\x}[1]{\textrm{#1}}
\newcommand{\dincludegraphics}{\includegraphics[width=0.5\textwidth]}
\newcommand{\tincludegraphics}{\includegraphics[width=0.33\textwidth]}

\title{CS70 - Lecture 15 Notes}
\author{Name: Felix Su$\quad$SID: 25794773}
\date{Spring 2016$\quad$GSI: Gerald Zhang}
\begin{document}
\maketitle

%%%% Topic %%%%
\subsection*{Review Turing and Halt}
%%%% Notes %%%%
\begin{itemize}
    \item Turing(P) includes Halt(P,P)
    \item If halts (infinite loop), if doesn't halt, halt (diagonalization)
    \item Halt program exists $\Rightarrow$ Turing program exists
    \item Turing("Turing") interpret program 'Turing' as text
    \begin{itemize}
        \item Neither halts nor loops (diagonlized statement on itself)
        \item Therefore, Turing program DNE
        \item No Turing program $\Rightarrow$ No Halt program (Contrapositive)
    \end{itemize}
\end{itemize}
%%%% Topic %%%%
\subsection*{Review Stars and Bars}
\href{https://en.wikipedia.org/wiki/Stars_and_bars_%28combinatorics%29}{Wikipedia: Stars and Bars}
%%%% Notes %%%%
\begin{itemize}
    \item Choose $n$ from $k$ with replacement, order doesn't matter
    \item Stars and Bars Bijection (Counting rule)
    \item Place $n$ (total) objects in $k$ bins
    \item $k$ bins are distinguishable, objects are not
    \item $k-1$ bars represent $k$ bins
    \item Thm 1 (positive nums): Each bin must contain an object, there can only be $\le 1$ bar between each star. You must choose $k-1$ bars from the $n-1$ available positions.
    \item Thm 2 (non-negative nums): Bins can contain any no. of objects, there can be $\ge k-1$ bars between each star. You must choose $k-1$ bars from the $n+(k-1)$ available positions.
\end{itemize}
\begin{mdframed}
\textbf{With Replacement/Order Doesn't Matter:}
\begin{align}\textrm{Positive Groups} &= \dbinom{n-1}{k-1}\end{align}
\begin{align}\textrm{Non-Negative Groups} &= \dbinom{n+(k-1)}{k-1}\end{align}
\end{mdframed}
%%%% Topic %%%%
\subsection*{Combinatorial Proofs}
%%%% Notes %%%%
\begin{mdframed}
\begin{itemize}
    \item Define what the Left side counts and what the Right side counts, then equate them
    \item Ask same question for both sides and answer them using the correct approach corresponding to the side
\end{itemize}
\end{mdframed}
\begin{itemize}
    \item $\binom{n}{k} = \binom{n}{n-k}$
    \begin{itemize}
        \item Left: Ways to choose $k$ out of $n$ items
        \item Right: Ways to (not) choose $n-k$ out of $n$ items, which is the same as choose $k$ out of $n$ items
    \end{itemize}
    \item $\binom{n}{k}=\binom{n-1}{k-1}+\cdots+\binom{k-1}{k-1}$
    \begin{itemize}
        \item Left: Ways to choose $k$ out of $n$ items
        \item Right: Sum number of subsets that include the first $i$ items and the number of subsets that do not include the first $i$ items
    \end{itemize}
    \item $2^n=\binom{n}{kn}+\binom{n}{n-1}+\cdots+\binom{n}{0}$ : \textbf{Binomial Thm}
    \begin{itemize}
        \item Sum of coefficients of an $(1+x)^n$ binomial ($n$th row in Pascal's Triangle)
        \item Left: No. of subsets of $n$ choices (element $i$ is either in or out of the subset, 2 poss.)
        \item Right: Sum of $\binom{n}{i}$ from $i$ to $n$
        \begin{itemize}
            \item $\binom{n}{i}$ ways to choose $i$ elements of $n$ choices
        \end{itemize}
    \end{itemize}
\end{itemize}
%%%% Topic %%%%
\subsection*{Pascal's Triangle}
%%%% Notes %%%%
\tincludegraphics{pascals}
\begin{itemize}
    \item Row $n$ = coefficients of $(1+x)^n$
    \item Choose $2^n$ terms: 1 or $x$ from $(1+x)$
    \begin{itemize}
        \item Combine all terms corresponding to $x^k$
        \item Coefficient of $x^k$ is $\binom{n}{k}$: you choose $k$ factors (products) that include $x$ and there are $n$ $x$'s to choose from
    \end{itemize}
\end{itemize}
\begin{mdframed}
\textbf{Pascal's Rule:}
\begin{itemize}
    \item Left: No. of $k$ subsets from $n+1$ choices
    \item Right: No. of subsets that choose the first item + No. of subsets that do not choose the first item = Left
    \begin{itemize}
        \item $\binom{n}{k-1}$: No. of subsets of $n$ that contain the first item. (Take away first item, left with $n$ items and $k-1$ remaining choices)
        \item $\binom{n}{k}$: No. of $k$ subsets of $n$ that do not contain the first item. (Take away first item, left with $n$ items, but still need to make $k$ choices.
    \end{itemize}
\end{itemize}
\begin{equation}\binom{n+1}{k}=\binom{n}{k}+\binom{n}{k-1}\end{equation}
\end{mdframed}
%%%% Topic %%%%
\subsection*{Simple Inclusion/Exclusion}
%%%% Notes %%%%
\begin{mdframed}
\begin{itemize}
    \item \textbf{Sum Rule: For disjoint sets} $S$ and $T$ : $|S\cup T|=|S|+|T|$
    \item \textbf{Inclusion/Exclusion Rule: For any} $S$ and $T$ : $|S\cup T=|S|+|T|-|S\cap T|$
    \begin{itemize}
        \item Ex. No. of 10 digit phone numbers that have 7 as the first or second digit
        \item $S$ = first digit 7. $|S|=10^9$
        \item $T$ = second digit 7. $|T|=10^9$
        \item $S\cap T$ = first and second digit. $|S\cap T|=10^8$
        \item $|S|+|T|-|S\cap T|=10^9+10^9-10^8$
    \end{itemize}
\end{itemize}
\end{mdframed}
\hrulefill
%%%% Topic %%%%
\subsection*{Probability Space}
%%%% Notes %%%%
\begin{itemize}
    \item Random Experiment: Define possible outcomes and likelihoods (percentages) have Statistical regularity
    \item Set of $\Omega$ outcomes: (Ex. $\Omega=\{H,T\}$)
    \begin{itemize}
        \item Probabilities assigned to each outcome: $\x{Pr}[H]=0.5, \x{Pr}[T]=0.5$
        \item Elements of $\Omega$ describes one outcome of the complete experiment
    \end{itemize}
    \item Assign probability to each outcome $\x{Pr}[A]$
    \begin{itemize}
        \item Probabilities assigned to each outcome: Ex. $\x{Pr}[H]=0.5, \x{Pr}[T]=0.5$
    \end{itemize}
\end{itemize}
\begin{mdframed}
\begin{itemize}
    \item $\Omega$ = sample space (can be countable or uncountable)
    \item $\omega \in \Omega$ = sample point
    \item probability $\x{Pr}[\omega]$ s.t. $0 \le \x{Pr}[\omega] \le 1$ and $\sum\limits_{\omega \in \Omega}\x{Pr}[\omega] = 1$
\end{itemize}
\end{mdframed}
\textbf{Uniform Probability Space}
\begin{itemize}
    \item Each outcome $\omega$ is equally probable: $\x{Pr}[\omega]=\frac{1}{\Omega}$ for all $\omega$
    \item $\Omega$ must be finite
\end{itemize}
\textbf{Non-Uniform Probability Space}
\begin{itemize}
    \item Each outcome $\omega$ is any $\x{Pr}[\omega]$ s.t. $0 \le \x{Pr}[\omega] \le 1$ and $\sum\limits_{\omega \in \Omega}\x{Pr}[\omega] = 1$
\end{itemize}
\end{document}