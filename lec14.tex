\documentclass{article}\usepackage{amsmath,amssymb,amsthm,tikz,tkz-graph,color,chngpage,soul,hyperref,csquotes,graphicx,floatrow, listings}\newcommand*{\QEDB}{\hfill\ensuremath{\square}}\newtheorem*{prop}{Proposition}\renewcommand{\theenumi}{\alph{enumi}}\usepackage[shortlabels]{enumitem}\usepackage[nobreak=true]{mdframed}\usetikzlibrary{matrix,calc}\MakeOuterQuote{"}\usepackage[margin=0.75in]{geometry} \newtheorem{theorem}{Theorem}\newcommand{\Z}{\mathbb Z}\newcommand{\R}{\mathbb R}\newcommand{\Q}{\mathbb Q}\newcommand{\N}{\mathbb N}

\title{CS70 - Lecture 14 Notes}
\author{Name: Felix Su$\quad$SID: 25794773}
\date{Spring 2016$\quad$GSI: Gerald Zhang}
\begin{document}
\maketitle

%%%% Topic %%%%
\subsection*{Counting}
%%%% Notes %%%%
\textbf{Tree Counting: Slow}
\begin{itemize}
    \item Build up string by bits, total amount of leaves is total possibilities
\end{itemize}
\begin{mdframed}
\textbf{First Rule of Counting: Product Rule:}
\begin{itemize}
    \item If objects constructed from a sequence of choices $n_1, n_2, ..., n_k$
    \item Total number of objects = $n_1 \times n_2 \times \cdots \times n_k$
\end{itemize}
\end{mdframed}

\textbf{Counting Functions/Polynomials}
\begin{itemize}
    \item There are $|T|^{|s|}$ functions $f : S \rightarrow T$
    \begin{itemize}
        \item $|T|$ choices for mapping of $f(s_i)$ (Use product rule)
    \end{itemize}
    \item $p^{d+1}$ polynomials of degree $d \bmod{p}$
    \begin{itemize}
        \item $p$ choices for each of the $d+1$ coefficients
    \end{itemize}
\end{itemize}

\textbf{Permutations}
\begin{itemize}
    \item Derived from the first rule of counting (product rule)
    \item Choose from less items each step
    \item Permutations of $n$ objects: number of orderings of $n$ objects (no replacements)
    \begin{itemize}
        \item $n \times (n-1) \times (n-2) \times\cdots \times 1 = n!$
    \end{itemize}
    \item Number of one to one functions $|S| \rightarrow |S|$ 
    \begin{itemize}
        \item Decreasing choices every step: $|S| \times |S|-1 \times \cdots \times 1 = |S|!$
    \end{itemize}
\end{itemize}
\begin{mdframed}
\textbf{Permutations}
\begin{itemize}
    \item Number of different samples of saize $k$ from $n$ numbers \textbf{without replacement}
    \begin{itemize}
        \item $n \times (n-1) \times (n-2) \times\cdots \times (n-(k-1)) = \frac{n!}{(n-k)!}$
    \end{itemize}
\end{itemize}
\end{mdframed}
\end{document}